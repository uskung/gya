\begin{abstract} 
    The double pendulum has long been considered a simple, yet well-defined chaotic system, which this report has intended to study by using numerical simulation. Therefore, Lagrangian mechanics were used to derive the equations of motions, which were numerically solved and simulated by using Eulers method and the Runge-Kutta method of the 4th order (the RK4 method). The double pendulum was then analyzed for a wide variety of starting angles, and it was shown that the RK4 method gave much more accurate simulations compared to Eulers method. Moreover, the simulations demonstrated the general chaotic nature of the double pendulum, which gave the conclusion that the level of chaos tended to increase as we increased the starting energy in the system.     
    
    % The double pendulum has always been a simple, yet well-defined chaotic system, which this report has intended to study. Therefore to carry out this , Lagrangian mechanics were used to derive the equations of motions, and then simulated with both Eulers Method, and the Runge-Kutta method of the fourth order.  

    % Using Lagrangian mechanics, we derived the equations of motion for the double pendulum, one of the simplest      
\end{abstract}