\pagenumbering{Roman}
\refstepcounter{chapter}
\chapter*{Appendix}
\appendix
\section*{A - Härledning av Euler-Lagrange ekvationen (\ref{eq:euler_lagrange_equation})} \label{appendix:euler_lagrange_equation}

Denna härledningen är huvudsakligen baserat på \cite[s. 222-223]{book:classical_mechanics:Morin2007}. Givet att funktionen $x_0(t)$ beskriver den väg av ett objekt, med startpunkterna $x(t_1) = x_1$ och $x(t_2) = x_2$, som ger ett stationärt\footnote{Kan alltså vara ett lokalt minimum, maximum eller terrasspunkt av $S$. Däremot är detta oftast ett minimum, därför det oftast benämns som \emph{principen om \emph{minsta} verkan.}} värde av objektets verkan $S$, då gäller att:
\begin{equation}
    \frac{d}{dt}\left( \frac{\partial \Lagr}{\partial \dot{q_{i}}} \right) - \frac{\partial \Lagr}{\partial q_{i}} = 0. 
    \label{eq:euler_lagrange_equation_appendix}
\end{equation}

\textit{Bevis:} Eftersom $x_0(t)$ är den funktion som ger ett stationärt värde av $S$, så kommer en funktion som är väldigt nära $x_0(t)$ ger i princip exakt samma verkan. Betrakta därmed funktionen:
\begin{equation}
    x_a(t) = x_0(t) + a\beta(t),
\end{equation}
där $a$ är ett tal och $\beta(t)$ uppfyller $\beta(t_1) = \beta(t_2) = 0$ så att ändpunkterna för $x_a(t)$ och $x_0(t)$ är detsamma. 

\section*{B - Kod för simulering av dubbelpendeln} \label{appendix:code}
dsanuhasd