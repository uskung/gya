\chapter*{Diskussion}
\refstepcounter{chapter}

Genom resultatet från simuleringarna av dubbelpendeln går det att besvara frågeställningarna: \emph{'Hur och vad är den bästa metoden för att simulera en dubbelpendel?'}, samt \emph{'Vilka slutsatser går det att dra utifrån dubbelpendelns kaotiska egenskaper?'}.  

Till en början genomfördes simuleringar för att undersöka vilken numerisk metod, Eulers stegmetod eller RK4-metoden, var bäst för att simulera en dubbelpendel. Detta resultat redovisades i figur \ref{fig:comparision_euler_rk4_small_angle} och \ref{fig:comparision_euler_rk4_large_angle}. I båda simuleringarna användes en steglängd på $h = 0.00005$ s. I figur \ref{fig:comparision_euler_rk4_small_angle}, där begynnelsevinklarna var relativt små ($\theta_1 = 0.5$, $\theta_2 = 0.5$ [rad]), började pendlarna att aldrig märkbart divergera. Detta går att förklara med att pendlarna aldrig började utveckla kaotiska inslag, utan pendeln fortsatte att svänga harmoniskt. Detta, till skillnad från dubbelpendlarna i figur \ref{fig:comparision_euler_rk4_large_angle}, där begynnelsevinklarna var relativt stora ($\theta_1 = 2.5$, $\theta_2 = 2.5$ [rad]), så divergerade pendlarna mycket fort. Redan i figur \ref{subfig:comparision_euler_rk4_large_angle_C}, när $t=4$ s, börjar pendlarna divergera märkbart, och i figur \ref{subfig:comparision_euler_rk4_large_angle_D} vid $t=6 s$ har pendlarna helt annorlunda färdvägar. 

