\chapter*{Diskussion}
\refstepcounter{chapter}

Genom resultatet från simuleringarna av dubbelpendeln går det att besvara frågeställningarna: \emph{'Hur och vad är den bästa metoden för att simulera en dubbelpendel?'}, samt \emph{'Vilka slutsatser går det att dra utifrån dubbelpendelns kaotiska egenskaper?'}.  

Till en början genomfördes simuleringar för att undersöka vilken numerisk metod, Eulers stegmetod eller RK4-metoden, var med lämpad för att simulera en dubbelpendel. Detta resultat redovisades i figur \ref{fig:comparision_euler_rk4_small_angle} och \ref{fig:comparision_euler_rk4_large_angle}. I båda simuleringarna användes en steglängd på $h = 0.00005$ s. I figur \ref{fig:comparision_euler_rk4_small_angle}, där begynnelsevinklarna var relativt små ($\theta_1 = 0.5$, $\theta_2 = 0.5$ [rad]), började pendlarna att aldrig märkbart divergera, även efter 60 sekunder. Detta går att förklara med att pendlarna aldrig började utveckla kaotiska inslag, utan pendeln fortsatte att svänga harmoniskt i sitt låga energitillstånd. Till skillnad från dubbelpendlarna i figur \ref{fig:comparision_euler_rk4_large_angle}, där begynnelsevinklarna var relativt stora ($\theta_1 = 2.5$, $\theta_2 = 2.5$ [rad]), så divergerade pendlarna mycket fort. Redan i figur \ref{subfig:comparision_euler_rk4_large_angle_C}, när $t=4$ s, börjar pendlarna divergera märkbart, och i figur \ref{subfig:comparision_euler_rk4_large_angle_D} vid $t=6 s$ har pendlarna tydligt helt annorlunda färdvägar. 

Detta visar tydligt hur känslig dubbelpendeln är till dess begynnelsevillkor, men även vikten av noggrannheten av dess simulering. Även om simuleringarna genomfördes med en relativt liten steglängd ($h = 0.00005$ s), så växer felet hos Eulers stegmetod så markant förhållandevis till RK4-metoden att Eulers-metoden blir praktiskt sätt oanvändbar. Endast vid låga energinivåer för pendeln, det vill säga när pendelns begynnelsevinklar är små, kan Eulers stegmetod vara användbar. Därmed utfördes resterande delar av alla simuleringar endast med RK4-metoden.

Anledningen till varför RK4-metoden fungerar så mycket bättre i detta fall är hur felet minskar relativt steglängden $h$. För Eulers stegmetod avtar felet proportionellt mot steglängden $h$, men för RK4-metoden avtar felet proportionellt mot steglängden $h^4$. Med andra ord, om vi halverade steglängden skulle felet för Eulers stegmetod halveras, medan för RK4-metoden skulle det minska med en faktor 16. Därmed blir det tydligt hur Eulers stegmetod fort divergerar relativt RK4-metoden. 

