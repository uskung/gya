\chapter*{Diskussion}
\refstepcounter{chapter}

Genom resultatet från simuleringarna av dubbelpendeln går det att besvara frågeställ-ningarna: \emph{'Hur och vad är den bästa metoden för att simulera en dubbelpendel?'}, samt \emph{'Vilka slutsatser går det att dra utifrån dubbelpendelns kaotiska egenskaper?'}.  

Till en början genomfördes simuleringar för att undersöka vilken numerisk metod, Eulers stegmetod eller RK4-metoden, var lämpligast för att simulera en dubbelpendel. Detta resultat redovisades i figur \ref{fig:comparision_euler_rk4_small_angle} och \ref{fig:comparision_euler_rk4_large_angle}, där båda simuleringarna använde en steglängd på $h = 0.00005$ s. I figur \ref{fig:comparision_euler_rk4_small_angle}, där begynnelsevinklarna var relativt små ($\theta_1 = 0.5$, $\theta_2 = 0.5$ [rad]), började pendlarna att aldrig märkbart divergera, även efter 60 sekunder. Detta går att förklara med att pendlarna aldrig började utveckla kaotiska inslag, utan att pendeln fortsatte att svänga periodiskt i sitt låga energitillstånd. Till skillnad från dubbelpendlarna i figur \ref{fig:comparision_euler_rk4_large_angle}, där begynnelsevinklarna var relativt stora ($\theta_1 = 2.5$, $\theta_2 = 2.5$ [rad]), så divergerade pendlarna mycket fort. Redan i figur \ref{subfig:comparision_euler_rk4_large_angle_C}, när $t=4$ s, började pendlarna divergera märkbart, och i figur \ref{subfig:comparision_euler_rk4_large_angle_D}, vid $t=6$ s, visade pendlarna tydligt helt annorlunda banor. 

Detta visar uppenbart hur känslig dubbelpendeln är till dess begynnelsevillkor, men även vikten av noggrannheten av dess simulering. Även om simuleringarna genomfördes med en relativt liten steglängd ($h = 0.00005$ s), så växer felet hos Eulers stegmetod så markant förhållandevis till RK4-metoden att Eulers-metoden blir praktiskt sätt oanvändbar. Endast vid låga energinivåer för pendeln, det vill säga när pendelns begynnelsevinklar är små, kan Eulers stegmetod vara användbar. Därmed utfördes resterande delar av alla simuleringar endast med RK4-metoden.

Orsaken till varför RK4-metoden fungerar så mycket bättre i detta fall är hur felet minskar relativt steglängden $h$. För Eulers stegmetod avtar felet proportionellt mot steglängden $h$, men för RK4-metoden avtar felet proportionellt mot steglängden $h^4$ \cite{mat290-runge-kutta-6.2}. Med andra ord, om vi halverar steglängden $h$, skulle felet för Eulers stegmetod halveras, medan för RK4-metoden skulle det minska med en faktor 16. Därmed blir det tydligt hur Eulers stegmetod fort divergerar relativt RK4-metoden. 

Figur \ref{fig:comparision_between_small_angle_perbutation} visar ett annat exempel på hur känsliga dubbelpendlar är för begynnelsevillkor. I simuleringen sattes den andra pendelns begynnelsevillkor till 0.00001\% större än den första pendeln (som hade startvillkoret $\theta_1 = 2.5$ och $\theta_2 = 2.5$ [rad]). Redan i figur \ref{subfig:comparision_between_small_angle_perbutation_D} ($t=6$ s) blir skillnaderna synliga mellan pendlarna, och sedan vid figur \ref{subfig:comparision_between_small_angle_perbutation_E} och \ref{subfig:comparision_between_small_angle_perbutation_F} ($t = 8$ respektive $t=10$ s) har pendlarnas bana helt divergerat. 

För att ge ett annat perspektiv på dubbelpendelns kaos plottades vinklarna $\theta_1$ och $\theta_2$ parametriskt mot varandra i figur \ref{fig:parametric_plots}, och sedan plottades vinklarna som funktion av tiden i figur \ref{fig:plots_as_function_of_time}. Dessa plottar visar att att dubbelpendelns kaos tenderar att följa unika mönster vid lägre energinivåer/begynnelsevinklar. I figurerna \ref{subfig:parametric_plots_A}, \ref{subfig:parametric_plots_D}, \ref{subfig:parametric_plots_E} och \ref{subfig:parametric_plots_F} visas olika mönster som bildats av dubbelpendelns simulering. Alla dessa pendlar vid lägre energinivåer/startvinklar visar former av olika periodiska svängningsrörelser och kommer aldrig att börja visa kaotiska beteenden. Varför de aldrig kommer bli kaotiska kan visas med hjälp av ett enkelt energiresonemang: Eftersom energin i systemet alltid kommer att bevaras (och att ingen energi tillförs), samt att pendlarnas lägesenergi vid startpositionen är så låg, kommer pendelns massor aldrig kunna rotera ett helt varv runt sin vridpunkt och därmed börja utveckla kaotiska beteenden. Att pendlarna följer en slags periodisk svängningsrörelse bekräftas även i figurerna \ref{subfig:plots_as_function_of_time_A}, \ref{subfig:plots_as_function_of_time_D}, \ref{subfig:plots_as_function_of_time_E} och \ref{subfig:plots_as_function_of_time_F}, där kurvorna tydligt visar sinusformade mönster. Att pendlar med små begynnelsevillkor aldrig kommer att divergera under längre tid demonstrerades även i figur \ref{fig:plots_of_stable_pendulum_over_long_time}, där en pendel simulerades under 180 s.

Simuleringar genomfördes även för pendlar med medelstora energinivåer/begynnelse-vinklar. I figur \ref{subfig:parametric_plots_C} visas en simulering med begynnelsevillkoret $\theta_1 = -1.5$ och $\theta_2 = 1.0$ [rad], där det syns att det parametriska mönstret som bildas är betydligt mindre väldefinierat än de med lägre energinivåer/begynnelsevinklar, men att $\theta_1$ och $\theta_2$ fortfarande stannar generellt inom samma område. Att pendeln inte följer ett lika väldefinierat mönster visas även av de mer oregelbundna sinuskurvorna i figur \ref{subfig:plots_as_function_of_time_E}.

Vid höga energinivåer/begynnelsevinklar visar simuleringarna att vinklarna $\theta_1$ och $\theta_2$ inte har något mönster alls. I figur \ref{subfig:parametric_plots_B} visas en pendel som hade startvinklarna $\theta_1 = 2.5$ och $\theta_2 = 2.5$ [rad]. Vinklarna börjar direkt bli till synes slumpartade, och kurvan i figur \ref{subfig:plots_as_function_of_time_B} visar en liknande divergering. Samma händer vid ännu längre simuleringar, som i figur \ref{subfig:plots_of_chaotic_pendulum_over_long_time_A} och \ref{subfig:plots_of_chaotic_pendulum_over_long_time_B} där pendlarna simulerades under 180 sekunder. 

Generellt sätt har simuleringarna visat att pendlarna tenderar att likna periodiska, sinusliknande mönster vid låga energinivåer/begynnelsevinklar, och kaotiska mönster vid höga energinivåer/begynnelsevinklar. Rimligtvis går det att ställa sig frågan: 'Var går gränsen till när en dubbelpendel blir kaotisk?'. 

Ett sätt att besvara detta är att plotta för vilka startvinklar dubbelpendeln gör så att dubbelpendeln vrider ett helt varv. Detta har bland annat Jeremy S. Heyl gjort \cite{heyl2008double}, se figur \ref{fig:fractal}. Där visas det vita som startpositioner där dubbelpendeln aldrig vänder ett helt fullt varv. Det som Heyl har beräknat ger även samma bild som tidigare: att vid högre energitillstånd/startvinklar tenderar pendlarna att bli mer kaotiska. Däremot visar hans graf att det även finns några energirika begynnelsepositioner som ger stabila pendlar. 

Därför demonstrerades även pendlarna i figur \ref{fig:plots_of_stable_pendulums} med startvinklarna $\theta_1 = 2.899$, $\theta_2 = 1.913$ samt $\theta_1 = 2.226$, $\theta_2 = -2.114$ [rad]. Figurerna visar att även fast pendlarna var relativt energirika, lyckas pendlarna ändå behålla stabila mönster. Detta bekräftas även i figurerna \ref{subfig:plots_of_stable_pendulums_C}, \ref{subfig:plots_of_stable_pendulums_D}, \ref{subfig:plots_of_stable_pendulums_E} och \ref{subfig:plots_of_stable_pendulums_F}, där de parametriska mönsterna som bildas både återupprepar sig, men även är periodiska. 

\begin{figure}[ht!]
    \centering
    \includegraphics[scale=0.35]{fractal.png}
    \caption{Fraktal över vilka begynnelsevinklar som ger att en dubbelpendel välter över, och därmed blir kaotisk. Färgkodningen i grafen visar hur fort pendeln välter över, där grön är relativt snabbt, och rött är relativt långsamt. För mer info och bildkälla, se \cite{heyl2008double}. }
    \label{fig:fractal}    
\end{figure}

\section{Slutsats}

Som svar till frågeställningarna visar resultaten att RK4-metoden är en betydligt effektivare metod för att simulera en dubbelpendeln. Dessutom visar resultaten att dubbelpendeln tenderar att visa mer kaotiska utslag vid större begynnelsevinklar, och mer periodiska mönster vid lägre begynnelsevinklar.  Detta bekräftas även av andra källor, bland annat av Jeremy S. Heyls artikel \cite{heyl2008double} och fraktal över vilka dubbelpendlar som är stabila respektive kaotiska. Dessutom påvisades mönster (främst vid låga energinivåer/begynnelsevinklar), både när vinklarna $\theta_1$ och $\theta_2$ plottades parametriskt mot varandra, men även när de plottades som funktion av tiden. 

\section{Framtidsutblick/vidare utveckling av dubbelpendeln}

