\chapter*{Inledning}

\section{Bakgrund}

Studiet av pendlar har länge vart en central del av fysikundervisningen. De flesta gymnasieelever (som har en gång studerat fysik) känner säkert igen att de flesta vanliga pendlarna kan beskrivas som en harmonisk svängningsrörelse. Inte minst känner säkert många gymnasiefysiker igen formeln för en pendels svängningstid, $T = 2\pi\sqrt{\frac{l}{g}}$. 

Tyvärr så ingår det ingen riktig fördjupning för pendlar inom gymnasiestudierna\footnote{Det går även att argumentera för motsatsen, det kanske är bättre att lämna det åt universitetsstudenter att lära sig\dots}, och mycket av det som lärdes ut om pendlar gäller bara om startvinkeln $\theta$ är relativt liten. Till exempel kan pendlar bara beskrivas som en harmonisk svängningsrörelse om startvinkeln $\theta$ är liten, och samma sak gäller för formeln för en pendels svängningstid.

Därmed kan en vanlig "enkel" pendel bli rätt så komplicerad, långt utanför gymnasiefysikens gränser. Däremot går det att utveckla problemet ännu mer; om två 




% den välkända formeln för en matematisk pendels svängningstid $T = 2\pi\sqrt{\frac{l}{g}}$, givet att pendeln svänger runt relativt små vinklar. 
% De flesta gymnasieelever som någon gång läst fysik kommer säkert ihåg att en vanlig pendel 
% Alla som har fysik under gymnasiet känner säkert igen pendlar. De kan relativt enkelt beskrivas med 



