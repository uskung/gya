\chapter*{Inledning}
\pagenumbering{arabic}
\setcounter{page}{1}
\addcontentsline{toc}{chapter}{Inledning} % optional, to include in ToC

\section{Introduktion och bakgrund}
Pendeln är något som alltid har fascinerat mänskligheten genom tiderna. Redan under det första århundradet lyckades de gamla kineserna utveckla en seismograf med hjälp av en pendel, vars funktion var att aktivera ett säkerhetssystem vid jordbävningar \cite{book:china_history}. Pendlar används dessutom än idag; Mora-klockors tidshållning bygger på en svängande pendel, medan den klassiska metronomens tickande styrs av en inverterad variant. Det är därmed tydligt hur pendeln är relevant än idag.

Det är därför inte förvånande att pendeln länge har varit en central del av fysikundervisningen, inte minst på gymnasiet. De flesta före detta naturvetenskapliga gymnasieelever känner säkert igen att majoriteten av de enkla pendlarna kan beskrivas som en harmonisk svängningsrörelse, och att formeln för en pendels svängningstid är $T = 2\pi\sqrt{\frac{l}{g}}$ \cite{halliday1997fundamentals}.

Tyvärr så ingår det ingen större fördjupning för pendlar inom gymnasiestudierna\footnote{Det går även att argumentera för motsatsen, det kanske är bättre att lämna det åt universitetsstudenter\dots}, och mycket av det som lärs ut om pendlar gäller bara om begynnelsevinkeln $\theta$ är relativt liten. Det är bara när detta villkor är uppfyllt som pendlar kan beskrivas som en approximativ harmonisk svängningsrörelse, och inte minst gäller detta även för svängningstidsformeln ovan \cite{halliday1997fundamentals}.

På så vis kan en till synes ''enkel'' pendel bli förhållandevis komplicerad, långt utanför gymnasiefysikens gränser. Det finns dock flera olika sätt att vidareutveckla problemet, bland annat går det att skapa en så kallad \emph{dubbelpendel} genom att koppla två enkla pendlar ihop. Mycket forskning har redan skett på dubbelpendeln, och idag används dubbelpendeln ofta som klassiskt exempel inom \emph{kaosteorin} \cite{britannica_chaos_theory_2025}. 

Kaosteori är nämligen det tvärvetenskapliga område inom fysiken och matematiken som studerar deterministiska system vars långsiktiga beteende blir praktiskt oförutsägbara, även om begynnelsevillkoren är kända till väldigt hög precision \cite{strogatz2018nonlinear}. En av de tidigaste pionjärerna inom kaosteori var metrologen Edward Lorenz, som år 1961 försökte simulera väderprognoser tillsammans med Ellen Fetter och Margaret Hamilton \cite{lorenz1963deterministic,Lorenz1960StatisticalPrediction}. Det var här som Lorenz och hans kollegor upptäckte att minimala skillnader i begynnelsevillkoren för simuleringen orsakade drastiskt olika väderprognoser. Lorenz, som blev väldigt fascinerat av atmosfärens kaotiska system, bestämde sig för att förenkla hans modeller till ett system av tre linjära ordinära differentialekvationer \cite{lorenz1963deterministic}. Lösningen till detta system är vad som grafiskt brukas kallas ''Lorenz-attraktorn'', se figur \ref{fig:lorenz_attractor}. Lorenz lyckades visa att denna attraktor kommer alltid att divergera och att lösningarna till ekvationssystemet aldrig kommer att återupprepa sig \cite{lorenz1963deterministic}. Dessutom, det som Lorenz blev mest känd för, var självaste formen av en fjäril i attraktorn. Detta, tillsammans med titeln på ett av hans tal\footnote{Ursprungstitel på engelska 'Predictability: Does the Flap of a Butterfly’s Wings in Brazil Set Off a Tornado in Texas?'. Talet hölls inför det 139:de årsmötet av American Association for the Advancement of Science (AAAS) \cite{lorenz1972butterfly}.} från 1972, är ursprunget till det populärvetenskapliga uttrycket ''fjärilseffekten'' -- att fladdrandet av en fjärils vinge i Brasilien kan orsaka en tromb i Texas\footnote{För vidare läsning om Lorenz attraktorn, se \cite{lorenz1963deterministic,Lorenz1960StatisticalPrediction}.}. 

\begin{figure}[h]
    \centering
    \includegraphics[scale=0.24]{Lorenz_attractor_yb.pdf}
    \caption{Exempel på en Lorenz-attraktor. Notera utseendet av en fjäril. Bildkälla: \emph{Lorenz system simulation}. Bild av Wikimol, 2005, Wikimedia Commons. \textbf{FRÅGA TILL ERIK: Hur ska man göra med källhänvisningar till bilder? Är det bättre att källhänvisa som vanligt med Bib\TeX (alltså att man får en siffra, precis som jag gjort med resterande källor), eller är det bättre att bara skriva källhanteringen här i bildtexten?}}
    \label{fig:lorenz_attractor}
\end{figure}

Även om en dubbelpendlar kanske inte orsakar tromber i Texas, gäller principen om fjärilseffekten och kaos fortfarande: dubbelpendlar är extremt känsliga för sina begynnelsevillkor. Många experiment har gjorts på dubbelpendlar, och ofta betraktas dubbelpendeln som en introduktion till kaotiska system. Därmed går det att analysera dubbelpendeln vidare, och frågan kvarstår om går att hitta något mönster i dess kaotiska natur. 

% Med detta menas att systemets rörelse är extremt känsligt för begynnelsevillkoren, det vill säga att extremt små skillnader i startpositionen ger gigantiska utslag i systemets rörelse, trots att den i varje ögonblick styrs av väldefinierade rörelseekvationer (CITE). 

% Kaosteori är nämligen det område inom fysiken och matematiken som studerar deterministiska system vars långsiktiga beteende blir praktiskt oförutsägbara (CITE), ett fenomen som brukar kallas för "fjärilseffekten" (CITE/FOOTNOTE).

%% På så sätt är det inte konstigt varför studiet av pendlar har länge varit en central del av fysikundervisningen, inte minst på gymnasiet. De flesta före detta (naturvetenskapliga) gymnasieeleverna känner säkert att de flesta enkla pendlarna kan beskrivas som en harmonisk svängningsrörelse, samt att formeln för en pendels svängningstid är  $T = 2\pi\sqrt{\frac{l}{g}}$. 

%% Tyvärr så ingår det ingen riktig fördjupning inom pendlar inom gymnasiestudierna, och en stor del av det som lärdes ut i gymnasieundervisningen gäller bara om startvinkeln $\theta$ är relativt liten. Det är bara när detta villkor är uppfyllt som pendlar kan beskrivas som en enkel harmonisk svängningsrörelse, eller att formeln för en pendels svängningstid ovan skulle gälla. 

%%På så vis kan en vanlig ''enkel'' pendel bli rätt så komplicerad, långt utanför gymnasiefysikens gränser. Däremot finns det många fler sätt att vidareutveckla problemet, bland annat går det att skapa en så kallad \emph{dubbelpendel} genom att koppla två enkla pendlar ihop. Mycket forskning har redan skett på dubbelpendeln, och idag används dubbelpendeln ofta som klassiskt exempel inom kaosteorin (CITE). Med detta menas att systemets rörelse är extremt känsligt för begynnelsevillkoren, det vill säga att extremt små skillnader i startpositionen ger gigantiska utslag i systemets rörelse, trots att den i varje ögonblick styrs av väldefinierade rörelseekvationer (CITE). Kaosteori är nämligen det område inom fysiken och matematiken som studerar deterministiska system vars långsiktiga beteende blir praktiskt oförutsägbara (CITE), ett fenomen som brukar kallas för "fjärilseffekten" (CITE/FOOTNOTE).


\section{Syfte och frågeställning}
Syftet med detta arbete är att undersöka dubbelpendelns kaotiska egenskaper. Genom att härleda dubbelpendelns rörelseekvationer med Lagrange-mekanik, kommer vi numeriskt simulera en dubbelpendel genom Eulers-stegmetod och Runge-Kuttametoden av fjärde ordningen, och sedan dra analyser utifrån dess egenskaper. Gymnasiearbetet har därmed gått ut på att besvara frågeställningarna: \emph{'Hur och vad är den bästa metoden för att simulera en dubbelpendel?'}, samt \emph{'Vilka slutsatser går det att dra utifrån dubbelpendelns kaotiska egenskaper?'}. 

\section{Avgränsning}
Detta arbete kommer bara beröra dubbelpendlar i två dimensioner, där all form av friktion och luftmotstånd försummas. Dessutom kommer alla simuleringar och analyser av dubbelpendlarna förenklas genom att bara ta hänsyn till startpositionerna där vinkelhastigheterna är noll, det vill säga att alla dubbelpendlar antas ha noll kinetisk energi vid början av simuleringen.  

% \section{Avgränsning}

% Arbetet kommer att avgränsas genom analysen av 

%Studiet av pendlar har länge vart en central del av fysikundervisningen. De flesta gymnasieelever (som gått ett naturvetenskapligt program) känner säkert igen att de flesta vanliga pendlarna kan beskrivas som en harmonisk svängningsrörelse. Inte minst känner säkert många gymnasiefysiker igen formeln för en pendels svängningstid, $T = 2\pi\sqrt{\frac{l}{g}}$. 
%Tyvärr så ingår det ingen riktig fördjupning för pendlar inom gymnasiestudierna\footnote{Det går även att argumentera för motsatsen, det kanske är bättre att lämna det åt universitetsstudenter att lära sig\dots}, och mycket av det som lärdes ut om pendlar gäller bara om startvinkeln $\theta$ är relativt liten. Det är bara när detta villkor är uppfyllt som pendlar kan beskrivas som en harmonisk svängningsrörelse, inte minst gäller detta även för svängningstidsformeln ovan.
%På så vis kan en vanlig ''enkel'' pendel bli rätt så komplicerad, långt utanför gymnasiefysikens gränser. Däremot finns det många fler sätt att vidareutveckla problemet, bland annat går det att skapa en så kallad \emph{dubbelpendel} genom att koppla två enkla pendlar ihop. Det visar sig att dubbelpendeln kan väldigt enkelt visa kaotiska beteenden och bli väldigt svår att förutspå rörelsen vid, givet att mätdatan inte är helt perfekt. Därmed demonstrerar en dubbelpendel inte minst klassisk dynamik, men det är också en tydlig tillämpning på kaosteori. 
% Till exempel kan pendlar bara beskrivas som en harmonisk svängningsrörelse om startvinkeln $\theta$ är liten, och samma sak gäller för formeln för en pendels svängningstid.
% den välkända formeln för en matematisk pendels svängningstid $T = 2\pi\sqrt{\frac{l}{g}}$, givet att pendeln svänger runt relativt små vinklar. 
% De flesta gymnasieelever som någon gång läst fysik kommer säkert ihåg att en vanlig pendel 
% Alla som har fysik under gymnasiet känner säkert igen pendlar. De kan relativt enkelt beskrivas med 
% The study of pendulums have long been a staple of upper secondary school physics cuticulae. Not only would most upper secondary school pupil recognize that most pendulums are harmonic, but they would probably also remember the classic formula for a pendulum's period: $T = 2\pi\sqrt{\frac{l}{g}}$. 
%Pendulums are a familiar object. They are found everywhere in our daily lives; everything from the ticking of a grandfather clock, to describing the motion of a kid on a swing. The metronome, an instrument which is essential for musicians, is formed on the principle of the inverted pendulum. 
% The so-called inverted pendulum forms the fundamental basis for the operation of a metronome, which uses this principle to maintain a steady rhythm through controlled oscillations 
% They are most often found in old grandfather clocks that chime timely every hour.  
% Pendulums have long been considered a staple of upper secondary school physics curricula. 