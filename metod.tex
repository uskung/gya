\chapter*{Numerisk simulering och metod}
\refstepcounter{chapter}

\section{Behandling av rörelseekvationerna inför simulering}
Som vi visade i sektion \ref{sec:härledning_av_dubbelpendeln}, ekvation (\ref{eq:equations_of_motion}), har vi härlett rörelseekvationerna för dubbelpendeln med hjälp av Lagrange-mekanik. Dessa är:
\begin{equation} \begin{cases}
        0 &= \left(m_1+m_2\right)l_1\ddot{\theta}_1 + m_2l_2\ddot{\theta}_2\cos(\theta_1-\theta_2)
        + m_2l_2\dot{\theta}_2^2\sin(\theta_1-\theta_2) \\
        &\quad + \left(m_1 + m_2\right)g\sin{\theta_1}. \\
        0 &= m_2l_2\ddot{\theta}_2 + m_2l_1\ddot{\theta}_1\cos(\theta_1 - \theta_2) - m_2l_1\dot{\theta}_1^2\sin(\theta_1 - \theta_2) + m_2g\sin{\theta_2}. \label{eq:equations_of_motion_numerical_section}
\end{cases} \end{equation}
För att däremot kunna använda (\ref{eq:equations_of_motion_numerical_section}) med Runge-Kuttametoden, måste de skrivas om till ordinära differentialekvationer på formen $x' = f(x,t)$. Vi börjar med att genomföra substitutionen $\dot{\theta}_1 = \omega_1$ och $\dot{\theta}_2$, vilket ger:
\begin{equation}
    \begin{cases}
        \dot{\theta}_1 &= \omega_1  \\
        \dot{\theta}_2 &= \omega_2 \\
        0 &= (m_1+m_2)l_1\dot{\omega}_1 + m_2l_2\dot{\omega}_2\cos(\theta_1 - \theta_2) + m_2l_2\omega_2^2\sin(\theta_1 - \theta_2) \\
        &\quad + (m_1+m_2)g\sin{\theta_1} \\
        0 &= m_2l_2\dot{\omega}_2 + m_2l_1\dot{\omega}_1\cos(\theta_1 - \theta_2) - m_2l_1\omega_1^2\sin(\theta_1-\theta_2) + m_2g\sin{\theta_2}. \label{eq:equations_of_motion_with_omega_numerical_section}
    \end{cases}
\end{equation}
Däremot är fortfarande tredje och fjärde ekvationerna i (\ref{eq:equations_of_motion_with_omega_numerical_section}) sammankopplade (på engelska, 'coupled'). För att kunna använda (\ref{eq:equations_of_motion_with_omega_numerical_section}) i Runge-kuttametoden gör vi följande substitution enligt (\ref{eq:greek_substitutions}). 
\begin{equation}
    \begin{cases}
        \Delta\theta &= \theta_1 - \theta_2 \\
        \alpha &= (m_1+m_2)l_1 \\
        \beta &= m_2l_2\cos(\Delta\theta) \\
        \gamma &= m_2l_1\cos(\Delta\theta) \\
        \delta &= m_2l_2 \\
        \epsilon &= -m_2l_2\omega_2^2\sin(\Delta\theta) - (m_1 + m_2)g\sin{\theta_1} \\
        \zeta &= m_2l_2\omega_1^2\sin(\Delta\theta) - m_2g\sin{\theta_2}
    \end{cases} \label{eq:greek_substitutions}
\end{equation}
Vi kan nu skriva om ekvationssystemet (\ref{eq:equations_of_motion_with_omega_numerical_section}) med hjälp av substitutionerna i (\ref{eq:greek_substitutions}) enligt:
\begin{equation}
    \begin{cases}
        \dot{\theta}_1 &= \omega_1  \\
        \dot{\theta}_2 &= \omega_2 \\
        0 &= \alpha\dot{\omega}_1 + \beta \dot{\omega}_2 - \epsilon \\
        0 &= \delta\dot{\omega}_2 + \gamma\dot{\omega}_1 - \zeta.
    \end{cases} \label{eq:equations_of_motion_greek}
\end{equation}
Vi kan nu skriva om tredje och fjärde ekvationerna i (\ref{eq:equations_of_motion_greek}) som ett system av ekvationer i matrisform enligt:
\begin{equation}
    \begin{pmatrix}
        \alpha & \beta \\
        \gamma & \delta \\
    \end{pmatrix}
    \begin{pmatrix}
        \dot{\omega}_1 \\
        \dot{\omega}_2
    \end{pmatrix}
    =
    \begin{pmatrix}
        \epsilon \\
        \zeta
    \end{pmatrix}\footnote{Detta eftersom matrismultiplikation ger att $\alpha\dot{\omega}_1 + \beta\dot{\omega}_2 = \epsilon$ och $\gamma\dot{\omega}_1 + \delta\dot{\omega}_2 = \zeta$, vilket är de ursprungliga ekvationerna.}
    \label{matrix:equations_of_motions_greek}
\end{equation}
Genom att multiplicera inversen av $\left(\begin{smallmatrix} \alpha & \beta \\ \gamma & \delta \end{smallmatrix}\right)$ i (\ref{matrix:equations_of_motions_greek}) får vi att:
\begin{align}
    \begin{pmatrix}
        \alpha & \beta \\
        \gamma & \delta \\
    \end{pmatrix}^{-1}
    \begin{pmatrix}
        \alpha & \beta \\
        \gamma & \delta \\
    \end{pmatrix}
    \begin{pmatrix}
        \dot{\omega}_1 \\
        \dot{\omega}_2
    \end{pmatrix}
    &=
    \begin{pmatrix}
        \alpha & \beta \\
        \gamma & \delta \\
    \end{pmatrix}^{-1}
    \begin{pmatrix}
        \epsilon \\
        \zeta
    \end{pmatrix} \notag \\ %new line here
    \begin{pmatrix}
        \dot{\omega}_1 \\
        \dot{\omega}_2
    \end{pmatrix}
    &=
    \frac{1}{\alpha\delta-\beta\gamma}
    \begin{pmatrix}
        \delta & -\beta \\
        -\gamma & \alpha
    \end{pmatrix}
    \begin{pmatrix}
        \epsilon \\
        \zeta
    \end{pmatrix} \notag \\ %new line here
    \begin{pmatrix}
        \dot{\omega}_1 \\
        \dot{\omega}_2
    \end{pmatrix}
    &= \frac{1}{\alpha\delta - \beta\gamma}
    \begin{pmatrix}
        \delta\epsilon - \beta\zeta \\
        -\gamma\epsilon + \alpha\zeta
    \end{pmatrix} \notag \\ %new line here
    \begin{pmatrix}
        \dot{\omega}_1 \\
        \dot{\omega}_2
    \end{pmatrix}
    &= 
    \begin{pmatrix}
        \frac{\delta\epsilon - \beta\zeta}{\alpha\delta - \beta\gamma} \vspace{1mm}\\
        \frac{\alpha\zeta - \gamma\epsilon}{\alpha\delta - \beta\gamma}
    \end{pmatrix} \label{matrix:separated_for_omega_1_omega_2}
\end{align}
Vi kan därmed avläsa från (\ref{matrix:separated_for_omega_1_omega_2}) nya, icke-sammankopplade uttryck för $\dot{\omega}_1$ och $\dot{\omega}_2$, vilket ger oss ekvationssystemet (\ref{eq:equations_of_motions_final}) som beskriver dubbelpendelns rörelse i ordinära, icke-sammankopplade differentialekvationer.
\begin{equation}
    \begin{cases}
        \dot{\theta}_1 &= \omega_1 \\
        \dot{\theta}_2 &= \omega_2 \\
        \dot{\omega}_1 &= \frac{\delta\epsilon - \beta\zeta}{\alpha\delta - \beta\gamma} \\
        \dot{\omega}_2 &= \frac{\alpha\zeta - \gamma\epsilon}{\alpha\delta - \beta\gamma}
    \end{cases} \label{eq:equations_of_motions_final}
\end{equation}
