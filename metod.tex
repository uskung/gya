\chapter*{Numerisk simulering och metod}
\refstepcounter{chapter}

\section{Teori}
\subsection{Runge-Kuttametoden (RK4-metoden)}
Differentialekvationerna i (\ref{eq:equations_of_motion}) är icke-linjära och av andra ordningen, vilket gör de omöjligt att lösa rent analystiskt. Därmed måste ekvationerna lösas numeriskt. Vid enklare differentialekvationer används \emph{Eulers stegmetod} vid numeriska lösningar av differentialekvationer. Nackdelen med denna metod är att lösningen fort divergerar från sin faktiska lösning. Därmed har \emph{Runge-Kuttametoden}\footnote{Egentligen är Runge-Kuttametoden är en familj av metoder, då den innefattar ett flertal metoder, bland annat Eulers stegmetod. Däremot brukas RK4-metoden menas med ''Runge-Kuttametoden'', vilket är den som användes här. För vidare förklaring, se \cite{Runge-Kutta_method/Butcher2008}. } (RK4-metoden) använts för att numeriskt lösa ekvationssystemet (\ref{eq:equations_of_motion})  \cite[s. 93]{Runge-Kutta_method/Butcher2008}. 

Runge-Kuttametoden är en metod som kan lösa ekvationer på formen $y' = f'(t, y)$, där $y(t)$ är den okända funktionen. Metoden bygger på att approximera lösningen stegvis med ett litet $h$, men istället för att bara beräkna nuvarande lutningen (det som sker i Eulers stegmetod), så beräknas flera mellanliggande lutningar, där ett viktad medelvärde istället dras.

\subsubsection{Definitionen av RK4-metoden}

Låt $\mathbf{X} = [x_1, x_2, \ldots, x_n]$ vara en tillståndsvektor som beskriver ett systems rörelse, där begynnelsevillkoret $\mathbf{X_0}$ är känt. Låt även $\mathbf{f} = [f_1, f_2, \ldots, f_n]$ vara en vektor med funktioner som beskriver systemets rörelseförändring, det vill säga $\dot{\mathbf{X}} = f(t,\mathbf{X})$. Därmed ger Runge-kuttametoden att nästa tillståndsvektor $\mathbf{X}_{n+1}$ definieras enligt:
\begin{equation}
    \mathbf{X}_{n+1} = \mathbf{X}_n + \frac{h}{6}\left(\mathbf{K}_{1n} + 2\mathbf{K}_{2n} + 2\mathbf{K}_{3n} + \mathbf{K}_{4n}\right)
\end{equation}
\begin{equation}
    t_{n+1} = t_n + h,
\end{equation}
där $\mathbf{K}_{1n}, \mathbf{K}_{2n}, \mathbf{K}_{3n}$ och $\mathbf{K}_{4n}$ är definierade sådan att:
\begin{align}
    \mathbf{K}_{1n} &= \mathbf{f}(t,\mathbf{x}_n) \\
    \mathbf{K}_{2n} &= \mathbf{f}\left(t + \frac{h}{2}, \mathbf{x}_n + \frac{h}{2}\mathbf{K}_{1n}\right) \\
    \mathbf{K}_{3n} &= \mathbf{f}\left(t + \frac{h}{2}, \mathbf{x}_n + \frac{h}{2}\mathbf{K}_{2n}\right) \\
    \mathbf{K}_{4n} &= \mathbf{f}\left(t + h, \mathbf{x}_n + h\mathbf{K}_{3n}\right).
\end{align}

\subsection{Behandling av rörelseekvationerna för RK4-metoden}
Som visades i sektion \ref{sec:härledning_av_dubbelpendeln}, ekvation (\ref{eq:equations_of_motion}), har rörelseekvationerna för dubbelpendeln härletts med hjälp av Lagrange-mekanik. Dessa är:
\begin{equation} \begin{cases}
        0 &= \left(m_1+m_2\right)l_1\ddot{\theta}_1 + m_2l_2\ddot{\theta}_2\cos(\theta_1-\theta_2)
        + m_2l_2\dot{\theta}_2^2\sin(\theta_1-\theta_2) \\
        &\quad + \left(m_1 + m_2\right)g\sin{\theta_1}. \\
        0 &= m_2l_2\ddot{\theta}_2 + m_2l_1\ddot{\theta}_1\cos(\theta_1 - \theta_2) - m_2l_1\dot{\theta}_1^2\sin(\theta_1 - \theta_2) + m_2g\sin{\theta_2}. \label{eq:equations_of_motion_numerical_section}
\end{cases} \end{equation}
För att däremot kunna använda (\ref{eq:equations_of_motion_numerical_section}) med Runge-Kuttametoden, måste de skrivas om till ordinära differentialekvationer på formen $y' = f(t,y)$. Därför genomfördes  substitutionen $\dot{\theta}_1 = \omega_1$ och $\dot{\theta}_2 = \omega_2$, vilket ger:
\begin{equation}
    \begin{cases}
        \dot{\theta}_1 &= \omega_1  \\
        \dot{\theta}_2 &= \omega_2 \\
        0 &= (m_1+m_2)l_1\dot{\omega}_1 + m_2l_2\dot{\omega}_2\cos(\theta_1 - \theta_2) + m_2l_2\omega_2^2\sin(\theta_1 - \theta_2) \\
        &\quad + (m_1+m_2)g\sin{\theta_1} \\
        0 &= m_2l_2\dot{\omega}_2 + m_2l_1\dot{\omega}_1\cos(\theta_1 - \theta_2) - m_2l_1\omega_1^2\sin(\theta_1-\theta_2) + m_2g\sin{\theta_2}. \label{eq:equations_of_motion_with_omega_numerical_section}
    \end{cases}
\end{equation}
Däremot är fortfarande tredje och fjärde ekvationerna i (\ref{eq:equations_of_motion_with_omega_numerical_section}) sammankopplade (på engelska, 'coupled'). För att kunna använda (\ref{eq:equations_of_motion_with_omega_numerical_section}) i Runge-kuttametoden gjordes följande substitution enligt (\ref{eq:greek_substitutions}). 
\begin{equation}
    \begin{cases}
        \Delta\theta &= \theta_1 - \theta_2 \\
        \alpha &= (m_1+m_2)l_1 \\
        \beta &= m_2l_2\cos(\Delta\theta) \\
        \gamma &= m_2l_1\cos(\Delta\theta) \\
        \delta &= m_2l_2 \\
        \epsilon &= -m_2l_2\omega_2^2\sin(\Delta\theta) - (m_1 + m_2)g\sin{\theta_1} \\
        \zeta &= m_2l_2\omega_1^2\sin(\Delta\theta) - m_2g\sin{\theta_2}
    \end{cases} \label{eq:greek_substitutions}
\end{equation}
Det går nu skriva om ekvationssystemet (\ref{eq:equations_of_motion_with_omega_numerical_section}) med hjälp av substitutionerna i (\ref{eq:greek_substitutions}) enligt:
\begin{equation}
    \begin{cases}
        \dot{\theta}_1 &= \omega_1  \\
        \dot{\theta}_2 &= \omega_2 \\
        0 &= \alpha\dot{\omega}_1 + \beta \dot{\omega}_2 - \epsilon \\
        0 &= \delta\dot{\omega}_2 + \gamma\dot{\omega}_1 - \zeta.
    \end{cases} \label{eq:equations_of_motion_greek}
\end{equation}
Därmed kan nu tredje och fjärde ekvationerna i (\ref{eq:equations_of_motion_greek}) skrivas om som ett system av ekvationer i matrisform enligt:
\begin{equation}
    \begin{pmatrix}
        \alpha & \beta \\
        \gamma & \delta \\
    \end{pmatrix}
    \begin{pmatrix}
        \dot{\omega}_1 \\
        \dot{\omega}_2
    \end{pmatrix}
    =
    \begin{pmatrix}
        \epsilon \\
        \zeta
    \end{pmatrix}\footnote{Detta eftersom matrismultiplikation ger att $\alpha\dot{\omega}_1 + \beta\dot{\omega}_2 = \epsilon$ och $\gamma\dot{\omega}_1 + \delta\dot{\omega}_2 = \zeta$, vilket är de ursprungliga ekvationerna.}
    \label{matrix:equations_of_motions_greek}
\end{equation}
Multiplikation med inversen av $\left(\begin{smallmatrix} \alpha & \beta \\ \gamma & \delta \end{smallmatrix}\right)$ i (\ref{matrix:equations_of_motions_greek}) ger:
\begin{align}
    \begin{pmatrix}
        \alpha & \beta \\
        \gamma & \delta \\
    \end{pmatrix}^{-1}
    \begin{pmatrix}
        \alpha & \beta \\
        \gamma & \delta \\
    \end{pmatrix}
    \begin{pmatrix}
        \dot{\omega}_1 \\
        \dot{\omega}_2
    \end{pmatrix}
    &=
    \begin{pmatrix}
        \alpha & \beta \\
        \gamma & \delta \\
    \end{pmatrix}^{-1}
    \begin{pmatrix}
        \epsilon \\
        \zeta
    \end{pmatrix} \notag \\ %new line here
    \begin{pmatrix}
        \dot{\omega}_1 \\
        \dot{\omega}_2
    \end{pmatrix}
    &=
    \frac{1}{\alpha\delta-\beta\gamma}
    \begin{pmatrix}
        \delta & -\beta \\
        -\gamma & \alpha
    \end{pmatrix}
    \begin{pmatrix}
        \epsilon \\
        \zeta
    \end{pmatrix} \notag \\ %new line here
    \begin{pmatrix}
        \dot{\omega}_1 \\
        \dot{\omega}_2
    \end{pmatrix}
    &= \frac{1}{\alpha\delta - \beta\gamma}
    \begin{pmatrix}
        \delta\epsilon - \beta\zeta \\
        -\gamma\epsilon + \alpha\zeta
    \end{pmatrix} \notag \\ %new line here
    \begin{pmatrix}
        \dot{\omega}_1 \\
        \dot{\omega}_2
    \end{pmatrix}
    &= 
    \begin{pmatrix}
        \frac{\delta\epsilon - \beta\zeta}{\alpha\delta - \beta\gamma} \vspace{1mm}\\
        \frac{\alpha\zeta - \gamma\epsilon}{\alpha\delta - \beta\gamma}
    \end{pmatrix} \label{matrix:separated_for_omega_1_omega_2}
\end{align}
Det går därmed att avläsa från (\ref{matrix:separated_for_omega_1_omega_2}) nya, icke-sammankopplade uttryck för $\dot{\omega}_1$ och $\dot{\omega}_2$, vilket ger ett ekvationssystemet (\ref{eq:equations_of_motions_final}) som beskriver dubbelpendelns rörelse i ordinära, icke-sammankopplade differentialekvationer.
\begin{equation}
    \begin{cases}
        \dot{\theta}_1 &= \omega_1 \\
        \dot{\omega}_1 &= \frac{\delta\epsilon - \beta\zeta}{\alpha\delta - \beta\gamma} \\
        \dot{\theta}_2 &= \omega_2 \\
        \dot{\omega}_2 &= \frac{\alpha\zeta - \gamma\epsilon}{\alpha\delta - \beta\gamma}
    \end{cases} \label{eq:equations_of_motions_final}
\end{equation}

\section{Numeriska metoder} \label{section:numeriska_metoder}
För att utnyttja Runge-kuttametoden definierades en tillståndsvektor \\$\mathbf{X} = [\theta_1, \omega_1, \theta_2, \omega_2]$, vilket därmed ger att $\dot{\mathbf{X}} = [\omega_1, \dot{\omega}_1, \omega_2, \dot{\omega}_2]$. Ett program skrevs sedan i Python med hjälp av dessa tillståndsvektorer, först med Eulers-stegmetod, sedan med Runge-Kuttametoden, för att simulera en dubbelpendel. Dessa jämfördes för att demonstrera effektiviteten av respektive metod. 

Sedan skrevs kod där två pendlar simulerades med RK4-metoden, där den andra pendelns hade 0.001\% större begynnelsevinklar relativt den första pendeln. Detta genomfördes för tre huvudsakliga energitillstånd\footnote{Det vill säga att de startades med olika vinklar, sådan att de får olika mängd potentiell energi. Ju större startvinklar, desto större blir potentiella energin i systemet} under 30 sekunder genom att starta vinklarna med vinklarna enligt tabell (TODO).

Därefter plottades vinklarna $\theta_1$ och $\theta_2$ parametriskt mot varandra enligt startvinklarna i tabell (TODO). Dessutom plottades vinklarna tillsammans mot varandra som funktion av tiden enligt samma värden i tabell (TODO). 

Sedan simulerades en dubbelpendel med RK4-metoden under 30 sekunder enligt startvinklarna i tabell (TODO), där vinklarna $\theta_1$ och $\theta_2$ plottades parametriskt mot varandra. Dessutom plottades även samma vinklar tillsammans som funktion av tiden, enligt samma begynnelsevinklar i tabell (TODO). 


För alla pendlar antogs att pendlarna släpptes från vila, det vill säga att $\omega_1$ och $\omega_2$ = 0 vid $t=0$ s. Dessutom antogs att båda massorna $m_1$, $m_2$ = 1 kg, längderna $l_1$, $l_2$ = 1 m, samt att $g = 9.82$ m/s$^2$. All simuleringskod som skrevs sammanställdes i appendix B.

% Dessutom skrevs kod som visar två pendlar samtidigt med nästan exakt samma startvillkor, men att vinklarna skiljer sig åt med en liten procentskillnad i storleksordningen $10^{-4}$. 

% Kod skrevs även för att plotta vinklarna $\theta_1$ och $\theta_2$ för pendlar som simulerades i 15 sekunder. Detta genomfördes för pendlar med olika startvinklar, alltså för pendlar med vinklar sådan att deras potentiella energi är olika. 

\begin{table}[h]
    \centering
    \begin{tabular}{||c|c|c||}
        \hline
        $\theta_1$ & $\theta_2$ & 'Energinivå' \\
        \hline
        0.5 & 0.5 & Låg\\
        1.6 & 1.6 & Medel\\
        2.5 & 2.5 & Hög\\
        \hline
    \end{tabular}
    \caption{Tre begynnelsevinklar för $\theta_1$ och $\theta_2$, där $\theta_1$ och $\theta_2$ motsvarar vinklarna i figur \ref{fig:double_pendulum_schematic}. Vinklarna har även angetts med respektive 'energinivå', det vill säga hur mycket potentiell energi som pendlarna har vid startläget.}
\end{table}
    
\begin{table}[h]
    \centering
    \begin{tabular}{||c|c||}
        \hline
        $\theta_1$ & $\theta_2$ \\
        \hline
        0.5 & 0.5 \\
        1.6 & 1.6 \\
        2.5 & 2.5 \\
        -0.5 & 0.3 \\
        2.119 & 1.623 \\
        2.899 & 1.913 \\
        2.9 & 3.1 \\
        -1.5 & 1.0 \\
        \hline
    \end{tabular}
    \caption{Flera olika begynnelsevinklar för $\theta_1$ och $\theta_2$, där $\theta_1$ och $\theta_2$ motsvarar vinklarna i figur \ref{fig:double_pendulum_schematic}. }
\end{table}