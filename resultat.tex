\chapter*{Resultat}
\refstepcounter{chapter}


Resultatet av simuleringarna som beskrevs i sektion \ref{section:numeriska_metoder} redovisades i figurerna nedan. Första simuleringen, där Eulers-stegmetod och Runge-Kuttametoden jämfördes, redovisades i figur (TODO). Simuleringen där två pendlar simulerades med RK4-metoden, där den andra pendeln hade 0.001\% större begynnelsevinklar, redovisades i figur (TODO). Resultatet från de parametriska plottarna av $\theta_1$ och $\theta_2$, samt plottarna av vinklarna $\theta_1$ och $\theta_2$ som funktion av tiden, redovisas i figur (TODO) respektive figur (TODO).   

\begin{figure}[ht]
    \begin{adjustwidth}{-2.5cm}{-2.5cm}
        \centering
        \begin{subfigure}[b]{0.45\linewidth}
            \centering
            \includegraphics[scale=0.55]{combined_euler_rk4_plot_at_t=0.0s_the1=0.5_the2=0.5.png}
            \caption{Startposition vid $t$ = 0 s.}
            \label{subfig:comparision_euler_rk4_small_angle_A}
        \end{subfigure}
        \hspace{-1cm}
        \begin{subfigure}[b]{0.45\linewidth}
            \centering
            \includegraphics[scale=0.55]{combined_euler_rk4_plot_at_t=60.0s_the1=0.5_the2=0.5.png}
            \caption{Position vid $t$ = 60 s.}
            \label{subfig:comparision_euler_rk4_small_angle_B}
        \end{subfigure}
    \end{adjustwidth}
    \caption{Jämförelse mellan Eulers stegmetod och RK4-metoden, den den blåa pendeln representerar Eulers stegmetod och den röda pendeln representerar RK4-metoden. Båda pendlarna hade startvinklarna $\theta_1 = 0.5$ rad och $\theta_2 = 0.5$ rad, enligt tabell \ref{tab:euler_rk4_comparision}. Eftersom metoderna aldrig divergerar märkbart går det inte att se den röda pendeln. Steglängden för båda metoderna var 0.00005 s. Streckade linjerna motsvarar färdvägen av pendeln de senaste två sekunderna}. 
    \label{fig:comparision_euler_rk4_small_angle}
\end{figure}

\begin{figure}[ht]
    \begin{adjustwidth}{-2.5cm}{-2.5cm}
        \centering
        \begin{subfigure}[b]{0.45\linewidth}
            \centering
            \includegraphics[scale=0.55]{combined_euler_rk4_plot_at_t=0.0s_the1=2.5_the2=2.5.png}
            \caption{Startposition vid $t$ = 0 s.}
            \label{subfig:comparision_euler_rk4_large_angle_A}
        \end{subfigure}
        \hspace{-1cm}
        \begin{subfigure}[b]{0.45\linewidth}
            \centering
            \includegraphics[scale=0.55]{combined_euler_rk4_plot_at_t=2.0s_the1=2.5_the2=2.5.png}
            \caption{Position vid $t$ = 2 s.}
            \label{subfig:comparision_euler_rk4_large_angle_B}
        \end{subfigure}

        \begin{subfigure}[b]{0.45\linewidth}
            \centering
            \includegraphics[scale=0.55]{combined_euler_rk4_plot_at_t=4.0s_the1=2.5_the2=2.5.png}
            \caption{Position vid $t$ = 4 s.}
            \label{subfig:comparision_euler_rk4_large_angle_C}
        \end{subfigure}
        \hspace{-1cm}
        \begin{subfigure}[b]{0.45\linewidth}
            \centering
            \includegraphics[scale=0.55]{combined_euler_rk4_plot_at_t=6.0s_the1=2.5_the2=2.5.png}
            \caption{Position vid $t$ = 6 s.}
            \label{subfig:comparision_euler_rk4_large_angle_D}
        \end{subfigure}
        \caption{Jämförelse mellan Eulers stegmetod och RK4-metoden, där den blåa pendeln representerar Eulers stegmetod och den röda pendeln representerar RK4-metoden. Båda pendlarna hade startvinklarna $\theta_1 = 2.5$ rad och $\theta_2 = 2.5$ rad enligt tabell \ref{tab:euler_rk4_comparision}. Vid simuleringen hade båda metoderna steglängden 0.00005 s. De streckade linjerna representerar färdvägen för pendeln under de senaste 2 sekunderna.}
        \label{fig:comparision_euler_rk4_large_angle}
    \end{adjustwidth}
\end{figure}

\begin{figure}[ht]
    \begin{adjustwidth}{-2.5cm}{-2.5cm}
        \vspace{-1.5cm}
        \centering
        \begin{subfigure}[b]{0.45\linewidth}
            \centering
            \includegraphics[scale=0.55]{multiple_pendulums_plot_at_t=0.0s_the1=2.5_the2=2.5_procent_diff=1.0000000000000001e-07.png}
            \caption{Startposition vid $t = 0$s.}
            \label{subfig:comparision_between_small_angle_perbutation_A}
        \end{subfigure}
        \hspace{-1cm}
        \begin{subfigure}[b]{0.45\linewidth}
            \centering
            \includegraphics[scale=0.55]{multiple_pendulums_plot_at_t=2.0s_the1=2.5_the2=2.5_procent_diff=1.0000000000000001e-07.png}
            \caption{Position vid $t = 2$s.}
            \label{subfig:comparision_between_small_angle_perbutation_B}
        \end{subfigure}

        \begin{subfigure}[b]{0.45\linewidth}
            \centering
            \includegraphics[scale=0.55]{multiple_pendulums_plot_at_t=4.0s_the1=2.5_the2=2.5_procent_diff=1.0000000000000001e-07.png}
            \caption{Position vid $t = 4$s.}
            \label{subfig:comparision_between_small_angle_perbutation_C}
        \end{subfigure}
        \hspace{-1cm}
        \begin{subfigure}[b]{0.45\linewidth}
            \centering
            \includegraphics[scale=0.55]{multiple_pendulums_plot_at_t=6.0s_the1=2.5_the2=2.5_procent_diff=1.0000000000000001e-07.png}
            \caption{Position vid $t = 6$s.}
            \label{subfig:comparision_between_small_angle_perbutation_D}
        \end{subfigure}

        \begin{subfigure}[b]{0.45\linewidth}
            \centering
            \includegraphics[scale=0.55]{multiple_pendulums_plot_at_t=8.0s_the1=2.5_the2=2.5_procent_diff=1.0000000000000001e-07.png}
            \caption{Position vid $t = 8$s.}
            \label{subfig:comparision_between_small_angle_perbutation_E}
        \end{subfigure}
        \hspace{-1cm}
        \begin{subfigure}[b]{0.45\linewidth}
            \centering
            \includegraphics[scale=0.55]{multiple_pendulums_plot_at_t=10.0s_the1=2.5_the2=2.5_procent_diff=1.0000000000000001e-07.png}
            \caption{Position vid $t = 10$s.}
            \label{subfig:comparision_between_small_angle_perbutation_F}
            
        \end{subfigure}
        \captionsetup{width=0.80\linewidth}
        \caption{Simulering av två pendlar med RK4-metoden. För den första pendeln (blåa) var begynnelsevinklarna $\theta_1 = 2.5$ rad och $\theta_2 = 2.5$ rad, medan för den andra pendeln (röda) var begynnelsevinklarna 0.00001\% större relativt den första pendeln. De streckade linjerna representerar färdvägen för pendlarna under de senaste 2 sekunderna.}
        \label{fig:comparision_between_small_angle_perbutation}
    \end{adjustwidth}
\end{figure}

\begin{figure}[ht]
    \vspace{-1.5cm}
    \begin{adjustwidth}{-2.5cm}{-2.5cm}
        \centering
        \begin{subfigure}[b]{0.45\linewidth}
            \centering
            \includegraphics[scale=0.55]{parametric_plot_the1=0.5_the2=0.5_t=30.0s.png}
            \caption{Startvinklar: $\theta_1 = 0.5$ och $\theta_2 = 0.5$ [rad].}
               \label{subfig:parametric_plots_A}
        \end{subfigure}
        \begin{subfigure}[b]{0.45\linewidth}
            \centering
            \includegraphics[scale=0.55]{parametric_plot_the1=2.5_the2=2.5_t=30.0s.png}
            \caption{Startvinklar: $\theta_1 = 2.5$ och $\theta_2 = 2.5$ [rad].}
               \label{subfig:parametric_plots_B}
        \end{subfigure}

        \begin{subfigure}[b]{0.45\linewidth}
            \centering
            \includegraphics[scale=0.55]{parametric_plot_the1=-1.5_the2=1.0_t=30.0s.png}
            \caption{Startvinklar: $\theta_1 = -1.5$ och $\theta_2 = 1.0$ [rad].}
            \label{subfig:parametric_plots_C}
        \end{subfigure}
        \begin{subfigure}[b]{0.45\linewidth}
            \centering
            \includegraphics[scale=0.55]{parametric_plot_the1=0.42_the2=-1.337_t=30.0s.png}
            \caption{Startvinklar: $\theta_1 = 0.42$ och $\theta_2 = -1.337$ [rad].}
               \label{subfig:parametric_plots_D}
        \end{subfigure}

        \begin{subfigure}[b]{0.45\linewidth}
            \centering
            \includegraphics[scale=0.55]{parametric_plot_the1=0.5_the2=0.3_t=30.0s.png}
            \caption{Startvinklar: $\theta_1 = 0.5$ och $\theta_2 = 0.3$ [rad].}
               \label{subfig:parametric_plots_E}
        \end{subfigure}
        \begin{subfigure}[b]{0.45\linewidth}
            \centering
            \includegraphics[scale=0.55]{parametric_plot_the1=-0.5_the2=0.3_t=30.0s.png}
            \caption{Startvinklar: $\theta_1 = -0.5$ och $\theta_2 = 0.3$ [rad].}
            \label{subfig:parametric_plots_F}
        \end{subfigure}
    \captionsetup{width=0.80\linewidth}
    \caption{Parametrisk plot av vinklarna $\theta_1$ och $\theta_2$ som funktion av tiden för dubbelpendlar som simulerades 30 s med RK4-metoden. Startvinklarna för $\theta_1$ och $\theta_2$ anges i respektive bildtext.}
    \label{fig:parametric_plots}
    \end{adjustwidth}
\end{figure}

\begin{figure}
    \vspace{-1.9cm}
    \begin{adjustwidth}{-2.5cm}{-2.5cm}
        \centering
        \begin{subfigure}[b]{0.45\linewidth}
            \centering
            \includegraphics[scale=0.55]{angles_as_functions_of_time_plot, the1=0.5_the2=0.5_t=20.0s.png}
            \caption{Startvinklar: $\theta_1 = 0.5$ och $\theta_2 = 0.5$ [rad]. Simuleringstid: 20s.}
            \label{subfig:plots_as_function_of_time_A}
        \end{subfigure}
        \hspace{0.5cm}
        \begin{subfigure}[b]{0.45\linewidth}
            \centering
            \includegraphics[scale=0.55]{angles_as_functions_of_time_plot, the1=2.5_the2=2.5_t=30.0s.png}
            \caption{Startvinklar: $\theta_1 = 2.5$ och $\theta_2 = 2.5$ [rad]. Simuleringstid: 30s.}
            \label{subfig:plots_as_function_of_time_B}
        \end{subfigure} 

        \begin{subfigure}[b]{0.45\linewidth}
            \centering
            \includegraphics[scale=0.55]{angles_as_functions_of_time_plot, the1=-1.5_the2=1.0_t=20.0s.png}
            \caption{Startvinklar: $\theta_1 = -1.5$ och $\theta_2 = 1.0$ [rad]. Simuleringstid: 20s.}
            \label{subfig:plots_as_function_of_time_C}
        \end{subfigure}
        \hspace{0.5cm}
         \begin{subfigure}[b]{0.45\linewidth}
            \centering
            \includegraphics[scale=0.55]{angles_as_functions_of_time_plot, the1=0.42_the2=-1.337_t=20.0s.png}
            \caption{Startvinklar: $\theta_1 = 0.42$ och $\theta_2 = -1.337$ [rad]. Simuleringstid: 20s.}
            \label{subfig:plots_as_function_of_time_D}
        \end{subfigure}

        \begin{subfigure}[b]{0.45\linewidth}
            \centering
            \includegraphics[scale=0.55]{angles_as_functions_of_time_plot, the1=0.5_the2=0.3_t=20.0s.png}
            \caption{Startvinklar: $\theta_1 = 0.5$ och $\theta_2 = 0.3$ [rad]. Simuleringstid: 20s.}
            \label{subfig:plots_as_function_of_time_E}
        \end{subfigure}
        \hspace{0.5cm}
        \begin{subfigure}[b]{0.45\linewidth}
            \centering
            \includegraphics[scale=0.55]{angles_as_functions_of_time_plot, the1=-0.5_the2=0.3_t=20.0s.png}
            \caption{Startvinklar: $\theta_1 = -0.5$ och $\theta_2 = 0.3$ [rad]. Simuleringstid: 20s.}
            \label{subfig:plots_as_function_of_time_F}
        \end{subfigure}
        \captionsetup{width=0.80\linewidth}
        \caption{Plot av vinklarna $\theta_1$ och $\theta_2$ som funktion av tiden. Alla dubbelpendlar simulerades i 20s, förutom i figur \ref{subfig:plots_as_function_of_time_B}, där den simulerades i 30 s. Simuleringen genomfördes med RK4-metoden.}
        \label{fig:plots_as_function_of_time}
    \end{adjustwidth}
\end{figure}

\begin{figure}[ht]
    \vspace{-2cm}
    \begin{adjustwidth}{-2.5cm}{-2.5cm}
        \centering
        \begin{subfigure}[b]{0.45\linewidth}
            \centering
            \includegraphics[scale=0.55]{parametric_plot_the1=2.9_the2=-3.1_t=180.0s.png}
            \captionsetup{width=0.9\linewidth}
            \caption{Parametrisk plot för $\theta_1$ och $\theta_2$ under 180 s. Startvillkor: $\theta_1 = 2.9$, $\theta_2 = -3.1$ [rad].}
            \label{subfig:plots_of_chaotic_pendulum_over_long_time_A}
        \end{subfigure}
        \begin{subfigure}[b]{0.45\linewidth}
            \centering
            \includegraphics[scale=0.55]{angles_as_functions_of_time_plot, the1=2.9_the2=-3.1_t=180.0s.png}
            \captionsetup{width=0.9\linewidth}
            \caption{Plot av $\theta_1$ och $\theta_2$ som funktion av tiden, totalt i 180 s. Startvillkor: $\theta_1 = 2.9, \theta_2 = -3.1$ [rad].}
            \label{subfig:plots_of_chaotic_pendulum_over_long_time_B}
        \end{subfigure}
        \caption{Plot av dubbelpendel med startvillkoren $\theta_1 = 2.9$ och $\theta_2 = -3.1$ [rad]. Simuleringen pågick under 180 s, och genomfördes med RK4-metoden med en steglängd på 0.00005 s.}
        \label{fig:plots_of_chaotic_pendulum_over_long_time}
    \end{adjustwidth}    
\end{figure}
\begin{figure}[h!]
    \begin{adjustwidth}{-3cm}{-3cm}
        \centering
        \begin{subfigure}[b]{\linewidth}
            \centering
            \includegraphics[scale=0.55]{angles_as_functions_of_time_plot, the1=0.7_the2=0.2_t=90.0s.png}
            \caption{Plot av $\theta_1$ och $\theta_2$ som funktion av tiden under 90 s.}
            \label{subfig:plots_of_stable_pendulum_over_long_time_A}
        \end{subfigure}
        \captionsetup{width=0.45\linewidth}
        \begin{subfigure}[b]{\linewidth}
            \centering
            \includegraphics[scale=0.55]{parametric_plot_the1=0.7_the2=0.2_t=180.0s.png}
            \caption{Parametrisk plot av $\theta_1$ och $\theta_2$ under 180 s. }
            \label{subfig:plots_of_stable_pendulum_over_long_time_B}
        \end{subfigure}
        \captionsetup{width=0.9\linewidth}
        \caption{Plot av dubbelpendel med startvillkoren $\theta_1 = 0.7$ och $\theta_2 = 0.2$ [rad]. Simuleringen genomfördes i 180 s, men för figur a plottades bara 90 s för att göra kurvan mer läsbar. }
        \label{fig:plots_of_stable_pendulum_over_long_time}
    \end{adjustwidth}
\end{figure}

\begin{figure}[ht]
    \begin{adjustwidth}{-2.5cm}{-2.5cm}
        \vspace{-2.4cm}
        \centering
        \begin{subfigure}[b]{0.45\linewidth}
            \centering
            \includegraphics[scale=0.55]{runge-kutta_plot_the1=2.899_the2=1.913_at_30.0s.png}
            \caption{Färdväg för dubbelpendeln. Startvillkor: $\theta_1 = 2.899$, $\theta_2 = 1.913$.}
            \label{subfig:plots_of_stable_pendulums_A}
        \end{subfigure}
        \hspace{0.5cm}
        \begin{subfigure}[b]{0.45\linewidth}
            \centering
            \includegraphics[scale=0.55]{runge-kutta_plot_the1=2.226_the2=4.169_at_30.0s.png}
            \caption{Färdväg för dubbelpendel. Startvillkor: $\theta_1 = 2.226$, $\theta_2 = -2.114$.}
            \label{subfig:plots_of_stable_pendulums_B}
        \end{subfigure}

        \begin{subfigure}[b]{0.45\linewidth}
            \centering
            \includegraphics[scale=0.55]{parametric_plot_the1=2.899_the2=1.913_t=30.0s.png}
            \caption{Parametrisk plot under 30 s av $\theta_1$ och $\theta_2$. Startvillkor: $\theta_1 = 2.899$, $\theta_2 = 1.913$. }
            \label{subfig:plots_of_stable_pendulums_C}
        \end{subfigure}
        \hspace{0.5cm}
        \begin{subfigure}[b]{0.45\linewidth}
            \centering
            \includegraphics[scale=0.55]{parametric_plot_the1=2.226_the2=-2.1141853071795866_t=30.0s.png}
            \caption{Parametrisk plot under 30 s av $\theta_1$ och $\theta_2$. Startvillkor: $\theta_1 = 2.226$, $\theta_2 = -2.114$.}
            \label{subfig:plots_of_stable_pendulums_D}
        \end{subfigure}

        \begin{subfigure}[b]{0.45\linewidth}
            \centering
            \includegraphics[scale=0.55]{angles_as_functions_of_time_plot, the1=2.899_the2=1.913_t=30.0s.png}
            \caption{Plot av vinklarna $\theta_1$ (blå) och $\theta_2$ (orange) som funktion av tiden. Startvillkor: $\theta_1 = 2.899$, $\theta_2 = 1.913$.}
            \label{subfig:plots_of_stable_pendulums_E}
        \end{subfigure}
        \hspace{0.5cm}
        \begin{subfigure}[b]{0.45\linewidth}
            \centering
            \includegraphics[scale=0.55]{angles_as_functions_of_time_plot, the1=2.226_the2=-2.1141853071795866_t=30.0s.png}
            \caption{Plot av vinklarna $\theta_1$ (blå) och $\theta_2$ (orange) som funktion av tiden. Startvillkor: $\theta_1 = 2.226$, $\theta_2 = -2.114$.}
            \label{subfig:plots_of_stable_pendulums_F}
        \end{subfigure}
        \captionsetup{width=0.80\linewidth}
        \caption{Simulering av pendlar med begynnelsevillkoren $\theta_1 = 2.899$ och $\theta_2 = 1.913$ [rad] (figur \ref{subfig:plots_of_stable_pendulums_A}, \ref{subfig:plots_of_stable_pendulums_C} och \ref{subfig:plots_of_stable_pendulums_E}), samt $\theta_1 = 2.226$ och $\theta_2 = -2.114$ [rad] (figur \ref{subfig:plots_of_stable_pendulums_B}, \ref{subfig:plots_of_stable_pendulums_D} och \ref{subfig:plots_of_stable_pendulums_F}). Figur \ref{subfig:plots_of_stable_pendulums_A} och \ref{subfig:plots_of_stable_pendulums_B} visar den streckade linjen den färdväg som pendeln tagit. Figur \ref{subfig:plots_of_stable_pendulums_C} och \ref{subfig:plots_of_stable_pendulums_D} visar en parametrisk plot för pendlarna. Figur \ref{subfig:plots_of_stable_pendulums_E} och \ref{subfig:plots_of_stable_pendulums_F} visar vinklarna $\theta_1$ och $\theta_2$ som funktion av tiden. Alla simuleringar genomfördes med RK4-metoden.}
        \label{fig:plots_of_stable_pendulums}
    \end{adjustwidth}
\end{figure}
