\chapter*{Resultat}
\refstepcounter{chapter}


Resultatet av simuleringarna som beskrevs i sektion \ref{section:numeriska_metoder} redovisades i figurerna nedan. Första simuleringen, där Eulers-stegmetod och Runge-Kuttametoden jämfördes, redovisades i figur (TODO). Simuleringen där två pendlar simulerades med RK4-metoden, där den andra pendeln hade 0.001\% större begynnelsevinklar, redovisades i figur (TODO). Resultatet från de parametriska plottarna av $\theta_1$ och $\theta_2$, samt plottarna av vinklarna $\theta_1$ och $\theta_2$ som funktion av tiden, redovisas i figur (TODO) respektive figur (TODO).   

\begin{figure}[ht]
    \begin{adjustwidth}{-2.5cm}{-2.5cm}
        \centering
        \begin{subfigure}[b]{0.45\linewidth}
            \centering
            \includegraphics[scale=0.55]{combined_euler_rk4_plot_at_t=0.0s_the1=0.5_the2=0.5.png}
            \caption{Startposition vid $t$=0 s.}
        \end{subfigure}
        \begin{subfigure}[b]{0.45\linewidth}
            \centering
            \includegraphics[scale=0.55]{combined_euler_rk4_plot_at_t=60.0s_the1=0.5_the2=0.5.png}
            \caption{Position vid $t$=60 s.}
        \end{subfigure}
    \end{adjustwidth}
    \caption{Jämförelse mellan Eulers stegmetod och RK4-metoden, den den blåa pendeln representerar Eulers stegmetod och den röda pendeln representerar RK4-metoden. Båda pendlarna hade startvinklarna $\theta_1 = 0.5$ och $\theta_2 = 0.5$, enligt tabell \ref{tab:euler_rk4_comparision}. Eftersom metoderna aldrig divergerar går det inte att se den röda pendeln. Streckade linjerna motsvarar färdvägen av pendeln de senaste två sekunderna}. 
\end{figure}

\begin{figure}[ht]
    \begin{adjustwidth}{-2.5cm}{-2.5cm}
        \centering
        \begin{subfigure}[b]{0.45\linewidth}
            \centering
            \includegraphics[scale=0.55]{combined_euler_rk4_plot_at_t=0.0s_the1=2.5_the2=2.5.png}
            \caption{Startposition vid $t$ = 0 s.}
        \end{subfigure}
        \begin{subfigure}[b]{0.45\linewidth}
            \centering
            \includegraphics[scale=0.55]{combined_euler_rk4_plot_at_t=2.0s_the1=2.5_the2=2.5.png}
            \caption{Position vid $t$ = 2 s.}
        \end{subfigure}

        \begin{subfigure}[b]{0.45\linewidth}
            \centering
            \includegraphics[scale=0.55]{combined_euler_rk4_plot_at_t=4.0s_the1=2.5_the2=2.5.png}
            \caption{Position vid $t$ = 4 s.}
        \end{subfigure}
        \begin{subfigure}[b]{0.45\linewidth}
            \centering
            \includegraphics[scale=0.55]{combined_euler_rk4_plot_at_t=6.0s_the1=2.5_the2=2.5.png}
            \caption{Position vid $t$ = 6 s.}
        \end{subfigure}
        \caption{Jämförelse mellan Eulers stegmetod och RK4-metoden, där den blåa pendeln representerar Eulers stegmetod och den röda pendeln representerar RK4-metoden. Båda pendlarna hade startvinklarna $\theta_1 = 2.5$ och $\theta_2 = 2.5$ enligt tabell \ref{tab:euler_rk4_comparision}. Vid simuleringen hade båda metoderna steglängden 0.00005 s. De streckade linjerna representerar färdvägen för pendeln under de senaste 2 sekunderna.}
    \end{adjustwidth}
\end{figure}